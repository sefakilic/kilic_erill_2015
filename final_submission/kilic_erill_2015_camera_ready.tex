\documentclass{llncs}
\usepackage[utf8]{inputenc}
\usepackage[hidelinks]{hyperref}
\usepackage{graphicx}
\usepackage{makeidx}  % allows for indexgeneration

\begin{document}

\title{Assessment of Transcription Factor Binding Motif and Regulon Transfer
  Methods}
\author{Sefa Kilic \and Ivan Erill}
\institute{University of Maryland Baltimore County\\
Department of Biological Sciences\\
1000 Hilltop Circle, Baltimore, Maryland 21250\\
\email{\{sefa1,erill\}@umbc.edu}}

\maketitle
\textbf{Introduction.}  Comparative genomics has been leveraged in many studies
to characterize transcriptional regulatory
networks~\cite{ravcheev2013genomic,meireles2009comparative}. However, despite
its fundamental importance in such studies, the effect of motif and regulon
transfer methods remains largely unstudied. Thanks in large part to
high-throughput experimental techniques, available experimental data has
increased dramatically over the last few years and it has become possible for
the first time to reliably assess methods used for regulatory network
reconstruction. In this study, we describe three different transfer methods that
define transcription factor (TF) binding motif in a target species given some
regulatory activity information in a reference species. Motif-based transfer is
performed using the reference binding motif to search for putative binding sites
in the target genome on the assumption that, for a given TF, the binding motif
is relatively well conserved across closely related species. This method has
been shown to perform well at inferring existing regulatory networks in
previously uncharacterized
genomes~\cite{leyn2013genomic,leyn2014comparative}. The alternative source of
prior information is the regulatory network itself. The putative regulon is then
constructed based on orthologous transfer of the reference regulon and
\textit{de novo} motif discovery is performed on the promoter regions of
putatively regulated target genes.

\textbf{Methods.} We compiled binding site data from several publicly available
databases including CollecTF~\cite{kilic2013collectf}, a database of
experimentally-validated sites. The first method that we tested is the direct
transfer using the collection of known binding sites from a model species to
build a position-specific scoring matrix (PSSM) which is used then to scan the
promoter regions of the target genome to identify putative sites. The second
method defines the motif by performing motif discovery on pre-searched candidate
sequences. After the PSSM search, promoters with high scoring sites are given as
input to the motif discovery algorithm with the motivation of capturing motifs
slightly different from the reference one and mitigating the effects of
inaccurate PSSM score threshold. The final method that we tested, called network
transfer, does not assume motif conservation. The underlying hypothesis is that
the regulon across two genomes might be functionally conserved to some degree
even if the binding motif is not. To define the motif in target species through
network transfer, the first step is to identify target regulon, the collection
of genes that are orthologous to the ones in the reference regulon. In the next
step, the promoters of operons in the target regulon are used for motif
discovery. To assess the performance of different transfer methods
quantitatively, we measured both (a) Euclidean distance between the true motif
and the inferred motif and (b) the area under ROC curve for the inferred
motif. To assess the significance of performances, we computed the distance and
area under ROC curve using a column-permuted version of the target motif as the
inferred motif.

\textbf{Results.}  We measured the performance of the transfer methods by
applying them to all pairs of species with at least 10 binding sites for a
particular TF, yielding 411 pairs of species belonging mostly to either Fur or
LexA. Our results show that direct transfer and motif discovery on pre-searched
promoters perform very similarly. Since these two methods are based on motif
conservation, they perform well when the TF proteins in the reference and target
species are highly similar. As the TF protein distance increases, their
performances decrease dramatically. Although network transfer is capable of
inferring conserved and non-conserved motifs for large protein distances in many
cases, our permutation analysis showed that, overall, the network transfer
method does not perform significantly well for any level of reference-target TF
distance. This finding is consistent with previous studies reporting high
plasticity in transcriptional regulatory
networks~\cite{price2007orthologous}. Another reason for poor network transfer
is the strictness of the orthology-based regulon transfer method. In the future,
we intend to relax the network transfer method by using functional similarity
(e.g., cluster of orthologous groups) for regulon transfer rather than direct
orthology. Also, we plan to investigate whether combining the information from
the extended network transfer with relaxed PSSM searches can enhance the
performance of direct transfer as the similarity between reference and target
motifs decays.

\bibliographystyle{splncs}
\bibliography{bibliography}

\end{document} 